\chapter{Micro-Optimization} \label{CHAP:MOPT}

\section{Counting bits}

\textbf{Question 1:} How many bits are set to one?

\textbf{Solution 1.1:} Simple approach 

Walk through every bit, if bit is one increase counter.

\begin{lstlisting}[language=C]
#define MAXBITS 32

int countBits(unsigned int x) { 
  int bits = 0; 
  for (int i = 0; i < MAXBITS; i++) { 
    if (((1<<i) & x) != 0) bits++; 
  } 
  return bits; 
}
\end{lstlisting}

\textit{Improvements:}

\begin{itemize}
\item Loop reversal
\begin{lstlisting}[language=C]
for (int i = MAXBITS-1; i>=0; i--)
\end{lstlisting}

\item Condition to computation
\begin{lstlisting}[language=C]
bits += (x >> i) & 1; 
\end{lstlisting}

\item Index removal
\begin{lstlisting}[language=C]
bits += (x & 1); 
x >>= 1; 
\end{lstlisting}

\item Loop inversion

\item Delay slot

\item Loop unrolling
\end{itemize}





\textbf{Solution 1.2:} Kerninghan bit clearing

\begin{lstlisting}[language=C]
int countBits(unsigned int x) { 
  int bits = 0; 
  while (x != 0) { 
    x = x & (x-1); 
    bits++; 
  } 
  return bits; 
}
\end{lstlisting}

\textbf{Solution 1.3:} Magic

\begin{lstlisting}[language=C]
int countBits(unsigned int x) { 
  return ((x * 0x000200040008001ULL) & 0x111111111111111ULL) % 0xf; 
}
\end{lstlisting}



\textbf{Question 2:} Is the number of bits set odd?

\textbf{Solution 2.1:} Halve \& Remove pairs
\begin{lstlisting}[language=C]
int bitCountIsOdd(unsigned int x) { 
  x ^= x >> 16; 
  x ^= x >> 8; 
  x ^= x >> 4; 
  x ^= x >> 2; 
  x ^= x >> 1; 
  return x & 1; 
}
\end{lstlisting}

\textbf{Solution 2.2:} Lookup table

Tradeoff between operation minimization and table size.
\begin{lstlisting}[language=C]
int bitCountIsOdd(unsigned int x) { 
  static unsigned char parity[] = { 
    0,1,1,0, 1,0,0,1, 
    1,0,0,1, 0,1,1,0 }; 
  x ^= x >> 16; 
  x ^= x >> 8; 
  x ^= x >> 4; 
  x &= 0xf; 
  return parity[x]; 
}
\end{lstlisting}

\textit{Improvement:} Inline bitmap
\begin{lstlisting}[language=C]
return (0x6996 >> x) & 1; 
\end{lstlisting}





\section{Block Parity}

Given n blocks of size SIZE each: Calculate block $n+1$ (d) such that every bit is the XOR of the matching bits in the other blocks.

\begin{lstlisting}[language=C]
#define SIZE 512

void blockParity(char *d, char **s, int n) { 
  for (int i = 0; i < SIZE; i++)
    d[i] = 0; 
  
  for (int j = 0; j < n; j++) { 
    for (int i = 0; i < SIZE; i++) {
      //Calculate block by block
      d[i] ^= s[j][i]; 
    }
  } 
}
\end{lstlisting}

\textit{Improvements:}
\begin{itemize}
\item Words\footnote{''A word is basically a fixed-sized group of digits that are handled as a unit by the instruction set or the hardware of the processor.'' (wikipedia)}
\item loop reversal
\item loop interchange
\item register (skip initialization)
\end{itemize}

\begin{lstlisting}[language=C]
#define WORDS 128

void blockParity(int *d, int **s, int n) { 
  for (int i = WORDS-1; i >= 0; i--) { 
    int t = s[0][i] ^ s[1][i];
    
    for (int j = 2; j < n; j++) {
      // Calculate bit by bit
      t ^= s[j][i]; 
    }
    
    d[i] = t; 
  } 
}
\end{lstlisting}

\textit{Further improvements:}

\begin{itemize}
\item Loop unrolling
\item Loop specialization
\end{itemize}

\begin{lstlisting}[language=C]
#define WORDS 128 

void blockParity(int *d, int **s, int n) { 
  switch (n) { 
    case 2: for (int i = WORDS-2; i>=0; i -= 2) 
              d[i] = s[0][i] ^ s[1][i];
              d[i+1] = s[0][i+1] ^ s[1][i+1]; 
            break; 
    case 3: for (int i = WORDS-2; i>=0; i -= 2) 
              d[i] = s[0][i] ^ s[1][i] ^ s[2][i];
              d[i+1] = s[0][i+1] ^ s[1][i+1] ^ s[2][i+1]; 
            break; 
    case ...
  } 
}
\end{lstlisting}

\begin{shadedSmaller}
\textbf{Summary:} Loop optimizations

\begin{description}
\item[loop reversal] \lstinline|i++ --> i--|
\item[index removal] \lstinline|p[i] --> p++|
\item[loop inversion] \lstinline|while --> do-while|
\item[loop unrolling] \lstinline|for(;;) x --> x;x;x;...|
\item[loop interchange] \lstinline|for(i;;) for(j;;) --> for(j;;) for(i;;)|
\item[loop unswitching] \lstinline|for(;;) if --> if for(;;) else for(;;)|
\end{description}
\end{shadedSmaller}

\begin{shadedSmaller}
\textbf{Summary:} Generic optimizations

\begin{description}
\item[condition to computation] \lstinline|if() x=... -> x=x+y| 
\item[lookup table] \lstinline|t[x]|
\item[inline bitmap] \lstinline|(t >> x) & mask|
\item[software pipelining] Group independent tasks, so that a processor with multiple instruction units can execute them in parallel. Do not confuse with hardware/processor pipelining. \lstinline|A(1)B(1)A(2)B(2) --> A(1)A(2)B(1)B(2)|
\item[instruction sharing] reuse instructions (e.g. by jump label)
\item[function call inlining] Replace function call with function body
\item[register usage] 
\end{description}
\end{shadedSmaller}

\textbf{int64x4 vs. int64x8} ???

\textbf{int64x4 vs. sse2x4} The int64 algorithms are cache-polluting\footnote{''Cache pollution describes situations where an executing computer program loads data into CPU cache unnecessarily.'' (wikipedia)}. This makes them slightly faster, but slowing the rest of the machine down. 




\section{One's complement checksum}

Used in IPv4 to calculate the header checksum.

\textit{Algorithm:} 
\begin{enumerate}
\item Add the first two 16-bit values
\item\label{carry} Add carry bit to result
\item Add the next 16-bit value
\item Go to \ref{carry}, until all values are added
\item Take one's complement (invert all bits)
\end{enumerate}

\section{Coffee shop}

Error-handling strategies for loosely coupled systems:

\begin{description}
\item[Write-off] Do nothing, or discard what you've done.

\item[Retry] Retry the one that failed.

\item[Compensating action] Undo the one that failed

\item[Transaction coordinator] Coordinator prepares action and commit only if both systems are ready.
\end{description}
