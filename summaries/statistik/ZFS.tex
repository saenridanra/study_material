%&&&&&&&&&&&&&&&&&&&&&&&&&&&&&&&&&&&&&&&&&%
%% Übungsblattvorlage					  %
%%% written by Andreas Rain				  %
%&&&&&&&&&&&&&&&&&&&&&&&&&&&&&&&&&&&&&&&&&%

\documentclass[fleqn, oneside, 10pt, titlepage]{article}
%
%Zeichensatz UTF-8 und T1-Zeichensatz
\usepackage[utf8x]{inputenc} 
\usepackage[T1]{fontenc}
%\newcommand{\changefont}[3]{\fontfamily{#1}\fontseries{#2}\fontshape{#3}\selectfont}
% 

%Silbentrennung nach neuer deutscher Rechtschreibung
\usepackage[ngerman]{babel} % Neue Rechtschreibung
%
%Einige Mathe-Symbole aus AMS-LaTeX
\usepackage{amsmath,amssymb,euscript}
%
% schönere Aufzählungen
\usepackage{enumerate}
%
% Für Bilder 
\usepackage{graphicx}
\usepackage{float} % lädt das Paket zur Verwendung von zusätzlichen Positionsbefehlen

%Graue box
\usepackage{framed}
\usepackage{xcolor}

\definecolor{MyBoxColor}{rgb}{0.9,0.9,0.9}
\newenvironment{shadedSmaller}{
  \def\FrameCommand{\fboxsep=\FrameSep \colorbox{MyBoxColor}}
  \MakeFramed {\advance\hsize-2\width\FrameRestore}}
{\endMakeFramed}

\newenvironment{shadedSmallerPadding}{
  \def\FrameCommand{\fboxsep=0.3cm \colorbox{MyBoxColor}}
  \MakeFramed {\advance\hsize-1.1\width\FrameRestore}}
{\endMakeFramed}
%%%%%%%%%%%%%%%%%%%%%%%%%%

% Standard Font auf Überschriften übernehmen.
%\usepackage{sectsty}
%\sectionfont{\fontfamily{pag}\fontseries{b}\fontsize{13pt}{20pt}\selectfont}
%\subsectionfont{\fontfamily{pag}\fontseries{b}\fontsize{11pt}{20pt}\selectfont}
%\subsubsectionfont{\fontfamily{pag}\fontseries{b}\fontsize{10pt}{20pt}\selectfont}
%\paragraphfont{\fontfamily{pag}\fontseries{b}\fontsize{11pt}{20pt}\selectfont}
%\subparagraphfont{\fontfamily{pag}\fontseries{b}\fontsize{10pt}{20pt}\selectfont}

%
% Farben und die Code-Umgebung einbinden
%
\usepackage{listings, color}
%
% Ein paar Standardfarben
\definecolor{darkblue}{rgb}{0,0,.6}
\definecolor{darkred}{rgb}{.6,0,0}
\definecolor{darkgreen}{rgb}{0,.6,0}
\definecolor{red}{rgb}{.98,0,0}
%
%Standardmäßig Java vorher laden
\lstloadlanguages{Java}
\lstloadlanguages{SQL}
% Standard-Layout für die Code-Umgebung (alle Sprachen)
\lstset{%
	language=SQL,
 	basicstyle=\footnotesize\ttfamily,
	showspaces=false,
	showtabs=false,
	columns=fixed,
	frame=none,
	numberstyle=\tiny,
	breaklines=true,
	showstringspaces=false,
	xleftmargin=0cm,
	tabsize=4,
	keywordstyle=\color{darkblue},
	commentstyle=\color{darkred},
	stringstyle=\color{darkgreen},
	emph={i, t, a, f},
	emphstyle=\color{red}
}%

%
% Seitenranddefinitionen
%   Links etwas mehr Rand als rechts, damit sich die Zettel später besser abheften lassen.
\usepackage[a4paper]{geometry} 
\geometry{a4paper,tmargin=1cm, bmargin=1cm, lmargin=1cm, rmargin=1cm, headheight=1em, headsep=1em, footskip=1cm} 

\usepackage{multirow}
\usepackage{rotating}
%Fancy Table
\usepackage{tabularx}
\usepackage{booktabs}
\usepackage{colortbl}
\usepackage{tikz}
\usetikzlibrary{calc}
\pgfdeclarelayer{background}
\pgfdeclarelayer{foreground}
\pgfsetlayers{background,main,foreground}
%

%
% Deutsche Absatzformatierung: Zwischen zwei Absätzen eine Zeile (1em) frei und 
% kein Einrücken der ersten Zeile
%
\setlength{\parskip}{1em}
\setlength{\parindent}{0pt}
\renewcommand{\arraystretch}{1.5}

%
% Beginn des Dokumentes
%
\begin{document}

\textbf{Lage bei ungruppierten Daten}\\
\scriptsize{
\begin{tabularx}{\textwidth}{p{13em}p{13em}p{13em}p{13em}p{10em}}
\textbf{Extremwerte}:\newline
	Maximum und Minimum
 &
\textbf{Modalwert: $mode(x)$}\newline
	Höchstes vorkommen eines Wertes
 &
\textbf{Median: $med(x)$}\newline
	$x_{\frac{n+1}{2}} \mid n \ mod \ 2 = 1$\newline
	$\left(x_{\frac{n}{2}} + x_{\frac{n}{2} + 1}\right) \div 2$, sonst
 &
\textbf{Quantile: $\tilde{x}_q$}\newline
 $\tilde{x}_q = [\lfloor n\cdot q\rfloor +1]$, falls $\tilde{x}_q \notin \mathbb{N}$\newline
 $\tilde{x}_q = \frac{[n \cdot q] + [n\cdot q + 1]}{2}$, sonst \newline
 Beachte, zweiter Fall $\longrightarrow$ Mittelwert!
 &
\textbf{Hinges/Eights:} \newline
	Tiefe(Med)=$(n+1)/2$ \newline
	Tiefe(Hinges)= \newline $([Tiefe(Med)]+1)/2$
	Falls nicht Ganzzahl, Mittelwert
 \\
\textbf{Arithmetisches Mittel: $\overline{x}$}\newline
	$\frac{1}{n}\cdot \sum\limits_{i=1}^n x_i$
 &
\textbf{q-getrimmtes Mittel:}\newline
	$\frac{1}{n - 2 \cdot \lfloor n\cdot q\rfloor}\cdot \sum\limits_{i=1+ \lfloor n\cdot q\rfloor}^{n-\lfloor n\cdot q\rfloor} x_i$
 &
\textbf{q-winsorisiertes Mittel:}\newline
 $\frac{1}{n} \left(\sum\limits_{i=1}^{\lfloor n\cdot q\rfloor} x_{\lfloor n\cdot q\rfloor + 1} + \sum\limits_{i=1}^{\lfloor n\cdot q\rfloor} x_{n - \lfloor n\cdot q\rfloor -1} + \sum\limits_{i=1+ \lfloor n\cdot q\rfloor}^{n-\lfloor n\cdot q\rfloor} x_i \right)$
 &
 
 &
\textbf{geometrisches Mittel: $\overline{x}_G$} \newline
	$\sqrt[n]{\prod\limits_{i=1}^n x_i}$
 \\ 
\end{tabularx} 
}

\textbf{Stetige Daten}\\
\scriptsize{
\begin{tabularx}{\textwidth}{p{13em}p{13em}p{13em}p{13em}p{10em}}
\textbf{Dichte:}\newline
	$\frac{p_i}{b_m}$\newline
	Wobei $m$ der Bereich ist und $b$ die breite.
 &
\textbf{Modalwert: $mode(x)$}\newline
	Höchste Dichte
 &
\textbf{Median: $med(x)$ \& Quantile}\newline
	$\tilde{x}_{q} = \dfrac{(q - p_{m-1}^k) \cdot b_m}{p_{m}^k - p_{m-1}^k}$
 &
\textbf{Arithmetisches Mittel: $\overline{x}$}\newline
	$\sum\limits_{i=1}^{m} \dfrac{x_{i,o} + x_{i,u}}{2} \cdot p_i$
 &
 \\ 
\end{tabularx} 
}

\normalsize
\textbf{Streuung / (Merkmalswert-)Dispersion}\\
\scriptsize{
\begin{tabularx}{\textwidth}{p{13em}p{13em}p{13em}p{13em}p{10em}}
\textbf{Spannweite: $sp$}\newline
	Maximum $-$ Minimum\newline
 &
\textbf{Quantilabstand: $d_q$}\newline
	$ \tilde{x}_{1-q} - \tilde{x}_{q}$
 &
\textbf{H-Spread $d_h$ / E-Spread $d_e$:}\newline
	Differenz zwischen oberem und unterem Hinge / Eight	
	
 &
\textbf{Mittlere Abweichung vom Median: $d_{\tilde{x}}$}\newline
	$\frac{1}{n} \sum\limits^n_{i=1} \vert x_i - \tilde{x} \vert$ 
 
 &
\textbf{Median aller Distanzen aller Werte zum Median: $MAD$} \newline
Med$(\vert x_i - \tilde{x} \vert)$ 
\\
\textbf{Varianz: $Var(x), s^2_n, s^2_{n-1}$}\newline
$sq = \sum\limits_{i=1}^n (x_1 - \overline{x})^2 $ für Urliste \newline
$sq = n\cdot\sum\limits_{i=1}^n (x_1 - \overline{x})^2 p_i $ für Verteilung \vspace{0.3em}\newline
$s^2 = \frac{sq}{n}$
&
\textbf{Varianz f. Linearkombination}\newline
$s^2_{a+b\cdot x}= b^2*s^2_x$\vspace{0.5em}\newline
\textbf{Standardabweichung}\newline
$Std(x) = \sqrt{s^2_n}, \sqrt{s^2_{n-1}}$\vspace{0.5em}\newline
\textbf{Standardfehler}\newline
$Std(\overline{x}) = Std(x) \div \sqrt{n}$
&
\textbf{Mittlere quad. Abweichung: $mqa$}\newline
$mqa = s^2_{n-1} \cdot 2$\vspace{0.5em}\newline
\textbf{Mittlere absolute Abweichung: $maa$}\newline
$\frac{2}{n\cdot(n-1)}\sum\limits_{i=1}^{n-1} i\cdot (n-1) \cdot (x_{i+1} - x_i)$
&
\textbf{Variationskoeffizient:}\vspace{0.3em}\newline
$v = \dfrac{s_x}{\overline{x}}$\vspace{0.5em}\newline
\textbf{Quartilsdispersions-koeffizient:}\vspace{0.3em}\newline
$v = \dfrac{\tilde{x}_.75 - \tilde{x}_.25}{\tilde{x}_.25 + \tilde{x}_.75}$
&
\textbf{Schiefe:}\vspace{0.4em}\newline
$schiefe(x)$\vspace{0.4em}\newline$ = \dfrac{\overline{x} - \tilde{x}}{s_x}$

\end{tabularx} 
}
\normalsize
\textbf{Maßzahlen für Dispersion}\\
\scriptsize{
\begin{tabularx}{\textwidth}{p{15em}p{15em}p{15em}p{15em}}
\textbf{Modaldispersion: $md$}\newline
	$md = 1-max(p_i)$, mehrere erlaubt.
 &
\textbf{Qualitative Varianz: $qv$}\newline
	$1 - \sum\limits_{i=1}^n p_i^2$
 &
\textbf{Entropie $h(x)$:}\newline
	$h(x)_b = -\sum\limits_{i=1}^I p_i \cdot log_b(p_i)$
&

\begin{tabular}{|c|c|c|}
\hline & $H_0$ akzept. & $H_0$ ablehnen  \\ 
\hline $H_0$ richtig & $1-\alpha$ & $\alpha$ (1. Fehler)  \\ 
\hline $H_a$ richtig & $\beta$ (2. Fehler) & $1-\beta$ \\ 
\hline 
\end{tabular} 
\end{tabularx} 
}
\normalsize
\textbf{Konfidenzintervall}\\
\scriptsize{
\begin{tabularx}{\textwidth}{p{15em}p{15em}p{15em}p{15em}}
\textbf{bekanntes $\sigma$ und $µ_0$ oder $\overline{x}$}\newline
	$o_{\alpha} = \overline{x} + z_{1-\frac{\alpha}{2}} \cdot \sigma_{\overline{x}}$\newline
	$u_{\alpha} = \overline{x} - z_{1-\frac{\alpha}{2}} \cdot \sigma_{\overline{x}}$\newline
	\textbf{Bestimmung des Intervalls}:\newline
	\textbf{\texttt{GTR:}}\texttt{STAT $\rightarrow$ TESTS $\rightarrow$ ZInterval}\newline
	Zweiseitig normal verfahren, $1-\alpha$ als \texttt{C-Level}. Einseitig $1-2\cdot\alpha$ als \texttt{C-Level} und nur eine Seite wählen.
	
 &
\textbf{$µ$ bei unbekanntem $\sigma$}\newline
	$o_{\alpha} = \overline{x} + t_{1-\frac{\alpha}{2}} \cdot s_{\overline{x}}$\newline
	$u_{\alpha} = \overline{x} - t_{1-\frac{\alpha}{2}} \cdot s_{\overline{x}}$\newline
	\textbf{Bestimmung des Intervalls}:\newline
	\textbf{\texttt{GTR:}}\texttt{STAT $\rightarrow$ TESTS $\rightarrow$ TInterval}\newline
	Zweiseitig normal verfahren, $1-\alpha$ als \texttt{C-Level}. Einseitig $1-2\cdot\alpha$ als \texttt{C-Level} und nur eine Seite wählen.
 &
\textbf{für Populationsanteil $\pi$:}\newline
für $n>100$: $\hat{\sigma}_{\overline{x}} = \sqrt{\frac{\pi(1-\pi)}{n}}$\newline
für $n<100$: $\hat{\sigma}_{\overline{x}} = \frac{\pi}{\sqrt{n}}$\newline
Standardabweichung is aus \textbf{SP} abgeschätzt. Wird jetzt in der Formel links verwendet.
&
\textbf{für Bootstrap-Methode:}\newline
Ziehen i-mal aus einer Stichprobe n Werte m.Z. \newline Für Ziehung $\rightarrow \overline{x} \& s_x$ als $\hat\theta$ in sortierter Liste. Dann ist \newline
$\hat\theta_u = \hat\theta^*_{\frac{a}{2}}$, $\hat\theta_o = \hat\theta^*_{1-\frac{a}{2}}$\newline
jeweils gesehen für $\overline{x} \& s_x$
\end{tabularx} 
}
\normalsize
\textbf{Hypothesentests}\\
\scriptsize{
\begin{tabularx}{\textwidth}{p{15em}p{15em}p{15em}p{15em}}
\textbf{Mittelwerttest bei bekanntem $\sigma_x$ (Z-Zest)}\newline
 Testwert: $Z = \dfrac{\overline{X}_n - µ_0}{\sigma_x} \sqrt{n}$\newline
 \textbf{\texttt{GTR:}}\texttt{STAT $\rightarrow$ TESTS $\rightarrow$ Z-Test}\newline
 Alle Werte eintragen, z ist der Testwert. Schauen ob in KB!\newline
 \textbf{Kritischer Bereich:}\newline
 Mit $µ_0$ im \texttt{GTR: ZInterval} ausfüllen.
 
 &
 \textbf{Mittelwerttest bei unbekanntem $\sigma_x$}\newline
 Testwert: $T = \dfrac{\overline{X}_n - µ_0}{s_{n-1}} \sqrt{n}$\newline
 \textbf{\texttt{GTR:}}\texttt{STAT $\rightarrow$ TESTS $\rightarrow$ T-Test}\newline
 Alle Werte eintragen, t ist der Testwert. Schauen ob in KB!\newline
 \textbf{Kritischer Bereich:}\newline
 Mit $µ_0$ im \texttt{GTR: TInterval} ausfüllen.
 &
 \textbf{Binomialtest}\newline
 $\alpha$:\newline \texttt{GTR $\rightarrow$ 2nd+DISTR $\rightarrow$ 1-binomcdf(n, p)}\newline
 \textbf{KB:} Ablesen in der Liste!\newline
 \textbf{Wahrscheinlichkeit} (z.B. weniger als 10 Brillenträger):\newline
 \texttt{GTR $\rightarrow$ 2nd+DISTR $\rightarrow$ 1-binomcdf(n, p, i)}\newline
 &
\textbf{Anpassungstest für Verteilungen}\newline
$LR_{\chi^2} = -2n \sum\limits_{i\in I} p_i\cdot ln\dfrac{\pi_i}{p_i}$\newline
$P_{\chi^2} = n \sum\limits_{i\in I} \dfrac{(p_i - \pi_i)²}{\pi_i}$\newline
\textbf{Freiheitsgrad $df$:}\newline
$df = (row - 1) \cdot (col-1)$\newline
\textbf{Kritischer Bereich:}\newline
In Tabelle für $df$ ablesen. \newline
Nullhypothese = $\pi_{ij} = \pi_{i.} \cdot \pi_{.j}$
\end{tabularx} 
}


\normalsize
\textbf{Verteilungen}\\
\scriptsize{
\begin{tabularx}{\textwidth}{p{15em}p{15em}p{15em}p{15em}}
\textbf{Normalverteilung $NV(µ, \sigma^2)$}\newline
 \textit{Wie groß ist n für Breite für gegebene} $NV(µ, \sigma^2)$?\newline
 $n = \left(z\cdot \dfrac{2 \cdot \sigma }{\mbox{KI-Breite}}\right)^2$, mit\newline
 $z$ aus der Standardnormalverteilung.\newline
 \textbf{Wahrscheinlichkeiten:}\newline
 $\Phi(-z)$, normalcdf mit $-z$ als upper value\& $-\infty$ für lower value\newline
 \textbf{Für binomial} ist $z = \dfrac{x-µ}{\sigma}$\newline
 \textbf{Für z.B. Mittelwert mit n} ist \newline $z = \dfrac{x-µ}{\frac{\sigma}{\sqrt{n}}}$\newline
 &
 \textbf{\textit{Beispiel:}} $NV(50,10\cdot10), n = 15$\newline
 $\pi$, sodass mehr als 5 größer 70?\newline
 1. z-standardi.:
 $z = \frac{70-50}{10} = 2$\newline
 2. $\pi = \Phi(-z) = 0.02275$\newline
 3. $P(k\geq 6)$ geht gegen 0.\newline
 \texttt{binomcdf(15,1-0,02275,15-6)}\newline\newline
 $\pi$, sodass genau 2 größer 80?\newline
 1. z-standardi.:
 $z = \frac{80-50}{10} = 3$\newline
 2. $\pi = \Phi(-z) = 0.00135$\newline
 3. $P(k = 2) = 0.000188$.\newline\newline
 \texttt{1} $-$
 	\texttt{bcdf(15,0,00135,2-1)}\newline
 $-$ \texttt{bcdf(15,1-0,00135,15-(2+1))}
 &
 $\pi$, sodass $µ$ 55?\newline
 1. z-standi.:
 $z = \frac{55-50}{10}\cdot \sqrt(10) = 1.58114$\newline
 2. $\pi = \Phi(-z) = 0.56923$\newline
 \newline
 &
 \normalsize\textbf{Rechenregeln, Erwartungswert, Varianz}\newline\scriptsize
 \textbf{Erwartungswert:}\newline
 Zufallsvariable X, $\sum X \cdot P(X)$
 \textbf{Varianz der Verteilung:}\newline
 $Var(X) = \left(\sum X^2 \cdot P(X)\right) - \left(\sum X \cdot P(X)\right)^2$
\end{tabularx} 
}
\newpage
\begin{tabularx}{\textwidth}{p{43em}p{23em}}
\begin{tabular}{p{1em} p{5em}|p{10em}|p{10em}|p{10em}|}
\cline{3-5}  & & \multicolumn{3}{|c|}{Mindestskalenniveau y-Merkmal} \\ 
\cline{3-5}  & & Nominal & Ordinal & Intervall \\ 
\cline{1-5} \multirow{4}{4mm}{\begin{sideways}Mindestskalenniveau x-Merkmal $\, \, \,$\end{sideways}} & Nominal & Kreuztabellenanalysen, Chi-
Quadrat, Nominale Korrelation
(Chi-Quadrat-Normierungen,
GOODMANs $\lambda$, KRUSKALs $\tau$,
kappa), Analyse mit loglinearen
und nominal-logistischen Modellen & Nichtparametrische Verfahren
(Mediantest,
WILCOXON-Tests, KRUSKAL-
WALLIS-Test, FRIEDMAN-
Varianzanalyse),
Analyse mit ordinallogistischen
Modellen & Vergleiche zweier Mittelwerte,
Varianzanalyse,
Vergleiche von Varianzen.
Nichtparametrische
Verfahren (bei Verletzung
von Verteilungs-
Voraussetzungen) \\ 
\cline{2-5}  & Ordinal & Analyse mit nominallogistischen
Modellen & Rangkorrelation
(SPEARMANs $\tau$, KENDALLs
$\rho$) & Mittelwerttests, Varianzanalyse \\ 
\cline{2-5}  & Intervall & Analyse mit nominallogistischen
Modellen (logistische
Regression), Univariate
Diskriminanzanalyse & Analyse mit ordinallogistischen
Modellen & PEARSON-Korrelation,
Regressionsanalyse \\ 
\cline{1-5} 
\end{tabular} 
&


\end{tabularx} 


\textbf{Wissensfragen:}\\
\scriptsize{
\begin{tabularx}{\textwidth}{p{13em}p{13em}p{13em}p{13em}p{10em}}
\textbf{Schätzer A, B, was bedeutet A ist effizienter als ?} $\rightarrow$ A hat geringere Varianz. \newline
\textbf{Welche Verteilung hat irgendein Parameter?} $\rightarrow$ Parameter können nicht verteilt sein. \newline
\textbf{Welchen Mangel hat der frequentistische Wahrscheinlichkeitsbegriff?} $\rightarrow$ Konvergenz nicht gewährleistet, zirkuläre Definition \newline 
\textbf{Was sind die Annahmen beim Varianzanalysemodell, die für das Testen erforderlich sind?} $\rightarrow$ Homoskedastizität für alle Gruppen, y mind. intervallskaliert, y in jeder Gruppe normalverteilt \newline
\textbf{Welche Möglichkeit gibt es, um die Macht des Test zu verbessern?} $\rightarrow$ Stichprobengröße erhöhen, $\alpha$ erhöhen, Hypothese anders formulieren, dass Unterschied zwischen $H_0$ und $H_a$ größer wird. \newline
\textbf{Welche Wahrscheinlichkeitsbegriffe wurden vorgestellt?} $\rightarrow$ Klassischer, frequentischer, subjektivistischer, axiomatischer. \newline
\textbf{Was besagt das Gesetz der großen Zahlen?} $\rightarrow$ Bei großen Stichproben liegt der Stichprobenmittelwert nahe des Populationsmittelwertes. \newline
\textbf{Was sind die Annahmen beim Regressionsmodell?} $\rightarrow$ Homoskedastizität, statistisch unabhängig. \newline
\textbf{Nennen sie zwei Unterschiede zwischen Fisher und Neyman/Pearson in Konzeption des Tests:} $\rightarrow$ F: $H_0$ niemals inhaltlich annehmbar. N/P: Fehler 1. und 2.Art. \newline 
\textbf{Wann sind zwei diskrete Zufallsvariablen X und Y stochastisch unabhängig?} $\rightarrow$ P(X) = P(X | Y). \newline
\textbf{Eigenschaften von Verteilung F(x):} $rightarrow$ Monoton nicht fallend, ganz links 0 und ganz rechts 1, Treppenfunktion. \newline
\textbf{Bei welchen Analysen wurde Homoskedastizität vorausgesetzt?} $rightarrow$ Regressionsanalyse, Varianzanalyse, T-Test für unverbundene Stichproben \newline
\textbf{Nennen sie die 3 Axiome von Kolmogoroff:} $\rightarrow$ Wahrscheinlichkeit zwischen 0 und 1, P(A oder B)= P(A) + P(B) falls sie disjunkt sind, Sicheres Ereignis: p=1. \newline

 &
\textbf{Aufgaben mit COV} \newline
Regressionsgerade: $ b = \dfrac{Cov(x,y)}{Var(x)}$,
$ a = \overline{y} - b \cdot \overline{x} $
$y = a + bx$ \newline
\textbf{Fehler-Regeln} \newline
$F(o):$\newline  $ssq(total) = (n-1) \cdot Var(Y)$ \newline
$F(m):$ \newline $  ssq(y,x) = (n-1) \cdot (Var(Y) - \dfrac{Cov(X,Y)^2}{Var(X)})$ \newline
\textbf{Dete rminationskoeffizient 2.Art} \newline
$r^2_{xy} = \dfrac{F(O) - F(M)}{F(O)}$ \newline $= \dfrac{Cov(X,Y)^2}{Var(X) \cdot Var(Y)}$ \newline
\textbf{Konfidenzinterfall für $\alpha$} \newline
$\hat{\alpha} \pm t_{0,95} (df) \cdot Stf(\alpha)$   mit \newline
$Stf(\alpha) = s_e \cdot \sqrt{\frac{1}{n} + \frac{\overleftarrow{x}^2}{(n-1) \cdot s^2_x}}$ \newline
$s_e = \sqrt{\frac{n-1}{n-2} \cdot (Var(Y) - \frac{Cov(X,Y)^2}{Var(X)}}$ \newline
$ \hat{\alpha}$ entspricht dem $a$ aus der Reg-Geraden. \newline 
$df = n-2$ (vielleicht) \newline 
\newline
\newline
\textbf{PRU-Maß bei Verteilungstabelle (X,Y)}\newline
x-bedingte Anteile in Tabelle aufstellen (Zeilenweiser Anteil) \newline
sowie Gesamtanteil für eine Zeile am Gesamten \newline
$Fehler_y(mit x) =  \sum p_i \cdot h_i(y),$ mit \newline
$h_i(y) = - \sum_j p_{ij} \cdot ln(p_{ij})$
$Fehler_y(ohne x) = - \sum_j p_{.j} ln(p_{.j})$, \newline
also Spaltenrandanteile als Y-Werte. \newline
$PRU = \frac{F(O) - F(M)}{F(O)}$, oder \newline
einfacher PRU: $PRU = \frac{\frac{LR_{\chi^2}}{2 \pi}}{F(O)}$
 \newline
 \newline
 
 &
 \textbf{Tabelle mit mehreren Stichprobenzusammenfassungen bei vorhanden $µ, s, n$}
\textbf{Determinationskoeffizient 1. Art \newline $\rho^2$:}
$\dfrac{ssq(total) -ssq(within}{ssq(total)}$ \newline 
$= \dfrac{ssq(between)}{ssq(within) + ssq(between)} $ \newline\newline
$ssq(between) =$\newline$ (\sum n_i \overline{y}_i^2) - n \overline{y}^2$ mit \newline\newline
$\overline{y} = \dfrac{\sum n_i \overline{y}_i^2}{n} $ \newline 
$ssq(within)$ \newline $ = \sum_i (n_i -1) s_i^2$ \newline \newline
Regel(mit X): $\hat{y}_{}ij = $ durchschnitt aller y-Mittelwerte
Fehler(mit X): $= ssq(within)$ \newline
$ssq(total) = ssq(between) + ssq(within)$ \newline
\newline
\textbf{Testen:}\newline
KB bestimmen und schauen ob Testwert in KB.\newline
TW: $F(df_1,df_2) = \frac{\eta^2}{1-\eta^2} \cdot \frac{df_2}{df_1}$ \newline
$df_1 = I-1, df_2 = n-I$
 &
 
 &

 \\

 &

 &

 &
 
 &

 \\ 
\end{tabularx} 
}




\end{document}